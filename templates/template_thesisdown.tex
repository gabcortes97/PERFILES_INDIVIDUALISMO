% This is the Reed College LaTeX thesis template. Most of the work
% for the document class was done by Sam Noble (SN), as well as this
% template. Later comments etc. by Ben Salzberg (BTS). Additional
% restructuring and APA support by Jess Youngberg (JY).
% Your comments and suggestions are more than welcome; please email
% them to cus@reed.edu
%
% See https://www.reed.edu/cis/help/LaTeX/index.html for help. There are a
% great bunch of help pages there, with notes on
% getting started, bibtex, etc. Go there and read it if you're not
% already familiar with LaTeX.
%
% Any line that starts with a percent symbol is a comment.
% They won't show up in the document, and are useful for notes
% to yourself and explaining commands.
% Commenting also removes a line from the document;
% very handy for troubleshooting problems. -BTS

% As far as I know, this follows the requirements laid out in
% the 2002-2003 Senior Handbook. Ask a librarian to check the
% document before binding. -SN

%%
%% Preamble
%%
% \documentclass{<something>} must begin each LaTeX document
\documentclass[12pt,twoside]{templates/facsothesis}
 \renewcommand{\familydefault}{\sfdefault}
% Packages are extensions to the basic LaTeX functions. Whatever you
% want to typeset, there is probably a package out there for it.
% Chemistry (chemtex), screenplays, you name it.
% Check out CTAN to see: https://www.ctan.org/
%%
\ifxetex
  \usepackage{polyglossia}
  \setmainlanguage{spanish}
  % Tabla en lugar de cuadro
  \gappto\captionsspanish{\renewcommand{\tablename}{Tabla}
          \renewcommand{\listtablename}{Índice de tablas}}
\else
  \usepackage[spanish,es-tabla]{babel}
\fi
%\usepackage[spanish]{babel}
\usepackage{graphicx,latexsym}
\usepackage{amsmath}
\usepackage{amssymb,amsthm}
\usepackage{longtable,booktabs,setspace}
\usepackage{chemarr} %% Useful for one reaction arrow, useless if you're not a chem major
\usepackage[hyphens]{url}
% Added by CII
%\usepackage{hyperref}
\usepackage[colorlinks = true,
            linkcolor = blue,
            urlcolor  = blue,
            citecolor = blue,
            anchorcolor = blue]{hyperref}
\usepackage{titlesec}
\titleformat{\chapter}[display]{\normalfont\bfseries}{}{0pt}{\Huge}
\titlespacing*{\chapter}{0pt}{-50pt}{12pt}
\usepackage{float}
\floatplacement{figure}{H}
% End of CII addition
\usepackage{rotating}
\usepackage{placeins} % para fijar la posición de las tablas con \FloatBarrier
\usepackage{helvet}

\usepackage[]{natbib}


% Next line commented out by CII
%\usepackage{biblatex}
%\usepackage{natbib}
% Comment out the natbib line above and uncomment the following two lines to use the new
% biblatex-chicago style, for Chicago A. Also make some changes at the end where the
% bibliography is included.
%\usepackage{biblatex-chicago}
%\bibliography{thesis}


% Added by CII (Thanks, Hadley!)
% Use ref for internal links
\renewcommand{\hyperref}[2][???]{\autoref{#1}}
\def\chapterautorefname{Chapter}
\def\sectionautorefname{Section}
\def\subsectionautorefname{Subsection}
% End of CII addition

% Added by CII
\usepackage{caption}
\captionsetup{width=5in}
% End of CII addition

% \usepackage{times} % other fonts are available like times, bookman, charter, palatino

% Syntax highlighting #22
$if(highlighting-macros)$
  $highlighting-macros$
$endif$

% To pass between YAML and LaTeX the dollar signs are added by CII
\title{$title$}
\author{$author$}
% The month and year that you submit your FINAL draft TO THE LIBRARY (May or December)
\date{$date$}
\division{$division$}
\advisor{$advisor$}
\institution{$institution$}
\degree{$degree$}
%If you have two advisors for some reason, you can use the following
% Uncommented out by CII
$if(altadvisor)$
\altadvisor{$altadvisor$}
$endif$
% End of CII addition

%%% Remember to use the correct department!
\department{$department$}
% if you're writing a thesis in an interdisciplinary major,
% uncomment the line below and change the text as appropriate.
% check the Senior Handbook if unsure.
%\thedivisionof{The Established Interdisciplinary Committee for}
% if you want the approval page to say "Approved for the Committee",
% uncomment the next line
%\approvedforthe{Committee}

% Added by CII
%%% Copied from knitr
%% maxwidth is the original width if it's less than linewidth
%% otherwise use linewidth (to make sure the graphics do not exceed the margin)
\makeatletter
\def\maxwidth{ %
  \ifdim\Gin@nat@width>\linewidth
    \linewidth
  \else
    \Gin@nat@width
  \fi
}
\makeatother

%Added by @MyKo101, code provided by @GerbrichFerdinands
$if(csl-refs)$
\newlength{\cslhangindent}
\setlength{\cslhangindent}{1.5em}
\newenvironment{cslreferences}%
  {$if(csl-hanging-indent)$\setlength{\parindent}{0pt}%
  \everypar{\setlength{\hangindent}{\cslhangindent}}\ignorespaces$endif$}%
  {\par}
$endif$

\setlength\parindent{0pt}


% Added by CII
$if(space_between_paragraphs)$
  %\setlength{\parskip}{\baselineskip}
  \usepackage[parfill]{parskip}
$endif$

\providecommand{\tightlist}{%
  \setlength{\itemsep}{0pt}\setlength{\parskip}{0pt}}



\Dedication{
$dedication$
}

\Preface{
$prefacio$
}

\Acknowledgements{
$agradecimientos$
}

\Abstract{
$abstract$
}

$for(header-includes)$
	$header-includes$
$endfor$

\renewcommand{\baselinestretch}{1.5}
% End of CII addition
%%
%% End Preamble
%%
%
\let\chaptername\relax
\begin{document}
\raggedbottom
\bibliographystyle{apa-good}
% Everything below added by CII
$if(title)$
  \maketitle
$endif$

\frontmatter % this stuff will be roman-numbered
 \pagestyle{empty} 

$if(prefacio)$
  \begin{prefacio}
  \thispagestyle{empty}
    $prefacio$
  \end{prefacio}
$endif$

$if(agradecimientos)$
  \begin{agradecimientos}
  \thispagestyle{empty}
  \setlength\parskip{1em plus 0.1em minus 0.2em}
    $agradecimientos$
  \end{agradecimientos}
$endif$

$if(abstract)$
  \begin{abstract}
  \thispagestyle{empty}
  \setlength\parskip{1em plus 0.1em minus 0.2em}
    $abstract$
  \end{abstract}
$endif$

$if(toc)$
%  \hypersetup{linkcolor=$if(toccolor)$$toccolor$$else$black$endif$}}
  \setcounter{tocdepth}{$toc-depth$}
  \setlength{\parskip}{0pt}
  \tableofcontents
  \thispagestyle{empty}
$endif$

\setlength\parskip{1em plus 0.1em minus 0.2em}

$if(lot)$
  \listoftables
  \thispagestyle{empty}
$endif$

$if(lof)$
  \listoffigures
  \thispagestyle{empty}
$endif$

$if(dedication)$
  \begin{dedication}
    $dedication$
  \end{dedication}
$endif$

\mainmatter % here the regular arabic numbering starts
\titleformat{\chapter}{\normalfont\Huge\bfseries}{\thechapter}{1em}{}
\pagestyle{fancyplain} % turns page numbering back on

$body$

% %%%%%%%%%%%%%%%%%%%%%%%%%%%%%%%%%%%%%%%%%%%%%%%%%
% %%% Bibliography                              %%%
% %%%%%%%%%%%%%%%%%%%%%%%%%%%%%%%%%%%%%%%%%%%%%%%%%
%\addtocontents{toc}{\vspace{.9\baselineskip}}

\pagestyle{fancyplain}
\fancyhf{}
	 \fancyhead[RE]{\slshape Bibliografía}
	 \fancyfoot[C]{\thepage}
\phantomsection
\addcontentsline{toc}{chapter}{Bibliografía}
\bibliography{tesis}

%% All books from our library (SfS) are already in a BiBTeX file
%% (Assbib). You can use Assbib combined with your personal BiBTeX file:
%% \bibliography{Myreferences,Assbib}. Of course, this will only work on
%% the computers at SfS, unless you copy the Assbib file
%%  --> /u/sfs/bib/Assbib.bib


\hypertarget{Anexo}{%
\chapter*{Anexo}\label{Anexo}}
\addcontentsline{toc}{chapter}{Anexo}

\begin{table}[!h]

\caption{\label{tab:unnamed-chunk-14}Indicadores Sociodemográficos por Perfiles de Individualismo}
\fontsize{8}{10}\selectfont
\begin{tabu} to \linewidth {>{\raggedright}X>{\raggedleft}X>{\raggedleft}X>{\raggedleft}X>{\raggedleft}X}
\toprule
\multicolumn{1}{c}{Indicador} & \multicolumn{1}{c}{Autoritario} & \multicolumn{1}{c}{Conservador} & \multicolumn{1}{c}{Liberal} & \multicolumn{1}{c}{Agéntico}\\
\midrule
\addlinespace[0.3em]
\multicolumn{5}{l}{\textbf{Edad}}\\
\hspace{1em}30 a 44 & 34,00 & 34,59 & 36,27 & 33,73\\
\hspace{1em}45 a 59 & 30,00 & 27,07 & 28,43 & 28,99\\
\hspace{1em}Mayores de 60 & 21,50 & 26,32 & 15,69 & 9,47\\
\hspace{1em}Menores de 30 & 14,50 & 12,03 & 19,61 & 27,81\\
\addlinespace[0.3em]
\multicolumn{5}{l}{\textbf{Género}}\\
\hspace{1em}Hombre & 51,50 & 48,87 & 50,49 & 44,97\\
\hspace{1em}Mujer & 48,50 & 51,13 & 49,51 & 55,03\\
\addlinespace[0.3em]
\multicolumn{5}{l}{\textbf{Identificación Política}}\\
\hspace{1em}Ninguna & 24,00 & 16,54 & 28,92 & 23,08\\
\hspace{1em}Izquierda & 4,50 & 0,00 & 8,33 & 7,10\\
\hspace{1em}Centro Izquierda & 21,00 & 23,31 & 19,61 & 24,26\\
\hspace{1em}Centro & 28,00 & 24,81 & 23,53 & 21,30\\
\hspace{1em}Centro Derecha & 21,00 & 35,34 & 15,20 & 21,30\\
\hspace{1em}Derecha & 1,50 & 0,00 & 4,41 & 2,96\\
\addlinespace[0.3em]
\multicolumn{5}{l}{\textbf{Ingresos Subjetivos}}\\
\hspace{1em}Ingresos Altos & 7,00 & 13,53 & 4,41 & 4,14\\
\hspace{1em}Ingresos Bajos & 19,50 & 15,79 & 17,65 & 13,61\\
\hspace{1em}Ingresos Medios-Altos & 28,50 & 18,05 & 26,47 & 24,85\\
\hspace{1em}Ingresos Medios-Bajos & 45,00 & 52,63 & 51,47 & 57,40\\
\addlinespace[0.3em]
\multicolumn{5}{l}{\textbf{Religion}}\\
\hspace{1em}Sin religión & 19,07 & 21,21 & 35,18 & 36,42\\
\hspace{1em}Católica & 67,01 & 62,88 & 55,28 & 53,70\\
\hspace{1em}Evangélica & 7,22 & 8,33 & 6,03 & 4,94\\
\hspace{1em}Otra & 6,70 & 7,58 & 3,52 & 4,94\\
\addlinespace[0.3em]
\multicolumn{5}{l}{\textbf{Tipo de Ciudad}}\\
\hspace{1em}Santiago & 44,00 & 28,57 & 38,73 & 36,09\\
\hspace{1em}Rural & 13,50 & 9,77 & 13,24 & 13,02\\
\hspace{1em}Menos de 100.000 & 17,00 & 26,32 & 11,76 & 14,20\\
\hspace{1em}Sobre 100.000 & 25,50 & 35,34 & 36,27 & 36,69\\
\addlinespace[0.3em]
\multicolumn{5}{l}{\textbf{Clase Social}}\\
\hspace{1em}Clase de Servicio & 26,16 & 23,93 & 23,30 & 20,67\\
\hspace{1em}Clase Media & 26,16 & 34,19 & 36,93 & 37,33\\
\hspace{1em}Clase Trabajadora & 47,67 & 41,88 & 39,77 & 42,00\\
\bottomrule
\multicolumn{5}{l}{\rule{0pt}{1em}\textit{Nota.} Porcentajes son relativos al total de las columnas}\\
\end{tabu}
\end{table}

\end{document}
